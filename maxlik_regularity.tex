%!TeX cleanPatterns = $OUTDIR/$JOB!($OUTEXT|.synctex.gz|.tex|.pdf), /$OUTDIR/_minted-$JOB/
\documentclass[12pt, a4paper]{article}

\usepackage{hyperref} % гиперссылки

\usepackage{physics}

\usepackage{tikz} % картинки в tikz
\usepackage{microtype} % свешивание пунктуации

\usepackage{array} % для столбцов фиксированной ширины

\usepackage{indentfirst} % отступ в первом параграфе

\usepackage{sectsty} % для центрирования названий частей
\allsectionsfont{\centering}

\usepackage{amsmath} % куча стандартных математических плюшек
\usepackage{amsthm} % теоремки
\usepackage{amssymb}

\usepackage{comment} % добавление длинных комментариев

\usepackage{bbm} % двойные буквы

\usepackage[top=2cm, left=1.2cm, right=1.2cm, bottom=2cm]{geometry} % размер текста на странице

\usepackage{lastpage} % чтобы узнать номер последней страницы

\usepackage{enumitem} % дополнительные плюшки для списков
%  например \begin{enumerate}[resume] позволяет продолжить нумерацию в новом списке

\usepackage{caption} % что-то делает с подписями рисунков :)

\usepackage{fancyhdr} % весёлые колонтитулы
\pagestyle{fancy}
\lhead{Условия регулярности правдоподобия}
\chead{}
\rhead{}
\lfoot{}
\cfoot{}
\rfoot{\thepage/\pageref{LastPage}}
\renewcommand{\headrulewidth}{0.4pt}
\renewcommand{\footrulewidth}{0.4pt}



\usepackage{todonotes} % для вставки в документ заметок о том, что осталось сделать
% \todo{Здесь надо коэффициенты исправить}
% \missingfigure{Здесь будет Последний день Помпеи}
% \listoftodos — печатает все поставленные \todo'шки



\usepackage{booktabs} % красивые таблицы
% заповеди из докупентации:
% 1. Не используйте вертикальные линни
% 2. Не используйте двойные линии
% 3. Единицы измерения - в шапку таблицы
% 4. Не сокращайте .1 вместо 0.1
% 5. Повторяющееся значение повторяйте, а не говорите "то же"



\usepackage{fontspec} % что-то про шрифты?
\usepackage{polyglossia} % русификация xelatex

\setmainlanguage{russian}
\setotherlanguages{english}

% download "Linux Libertine" fonts:
% http://www.linuxlibertine.org/index.php?id=91&L=1
\setmainfont{Linux Libertine O} % or Helvetica, Arial, Cambria
% why do we need \newfontfamily:
% http://tex.stackexchange.com/questions/91507/
\newfontfamily{\cyrillicfonttt}{Linux Libertine O}

\AddEnumerateCounter{\asbuk}{\russian@alph}{щ} % для списков с русскими буквами
\setlist[enumerate, 2]{label=\asbuk*),ref=\asbuk*}

%% эконометрические сокращения
\DeclareMathOperator{\Cov}{Cov}
\DeclareMathOperator{\Corr}{Corr}
\DeclareMathOperator{\Var}{Var}
\DeclareMathOperator{\E}{E}
\DeclareMathOperator{\Supp}{Supp}

\usepackage[bibencoding = auto,
backend = biber,
sorting = none,
style=alphabetic]{biblatex}

\addbibresource{maxlik_regularity.bib}

\newcommand{\RR}{\mathbb{R}}
\newcommand{\cN}{\mathcal{N}}
\DeclareMathOperator{\Med}{Med}



% делаем короче интервал в списках
\setlength{\itemsep}{0pt}
\setlength{\parskip}{0pt}
\setlength{\parsep}{0pt}

\newtheorem{theorem}{Теорема}

\newtheorem{assumption}{Предпосылка}



\begin{document}

\section*{Предпосылки}

\begin{assumption}[IID] 
  Случайные величины $X_1$, $X_2$, \ldots, $X_n$ независимы и одинаково распределены.
\end{assumption}
  

\begin{assumption}[Density] 
  У величины $X_i$ есть функция плотности $f_{\theta}(x)$.
\end{assumption}

\begin{assumption}[Identifiability] 
  Если $\theta \neq \theta'$, то законы распределения $f_\theta$ и $f_{\theta'}$ отличаются.
\end{assumption}


\begin{assumption}[Support] 
  Носитель $\Supp f_\theta$ не зависит от $\theta$. 
  Носителем называют замыкание множества, на котором функция плотности положительна.
\end{assumption}


\begin{assumption}[Interior] 
  Истинное $\theta_T$ является внутренней точкой множества всех возможных значений неизвестного параметра $\Theta$.
\end{assumption}

\begin{assumption}[Differentiable-k] 
  Плотность $f_\theta(x)$ дифференциируема по $\theta$ как минимум $k$ раз.
\end{assumption}

\begin{assumption}[Uniqueness] 
  Условие первого порядка $\ell'(\theta)=0$ имеет единственное решение. 
\end{assumption}

\begin{assumption}[Interchange] 
  Интеграл $I = \int_{\RR} f_\theta (x) dx$ дважды дифференцируем по $\theta$ и 
  вторая производная может быть найдена путём смены порядка интегрирования и дифференцирования.
\end{assumption}

\begin{assumption}[Bound] 
  Существует функция $M(x)$, такая что 
  $\E_T(M(X_i)) < \infty$ и $\abs{\partial^3 \ln f_\theta(x)/\partial \theta^3} \leq M(x)$ для всех $x$ 
  и всех $\theta$ из некоторой открытой окрестности настоящего $\theta_T$.
\end{assumption}

\section*{Теоремы}

\begin{theorem}
  Если выполнены условия [IID], [Density], [Identifiability], [Support], [Interior], [Differentiable-1], [Uniqueness], то 
  последовательность оценок максимального правдоподобия $(\hat \theta_n)$ состоятельная. 
\end{theorem}

\begin{theorem}
  Если выполнены условия [IID], [Density], [Identifiability], [Support], [Interior], [Differentiable-2], [Interchange], то 
  \[
  \Var(\hat \theta_n) \geq \frac{(d \E(\hat\theta_n) / d\theta )^2}{I_F}  
  \]
\end{theorem}

\begin{theorem}
  Если выполнены условия [IID], [Density], [Identifiability], [Support], [Interior], [Differentiable-3], [Interchange], [Bound], то 
  оценки асимптотически нормальны
  \[
     \sqrt{I_F} (\hat\theta_n - \theta_T) \to \cN(0;1)
  \]
  и, в частности, асимптотически эффективны:
  \[
  \lim_{n\to\infty} \Var(\hat \theta_n) I_F = 1  
  \]
\end{theorem}


\section*{Фабулы доказательств}

\section*{Формальные доказательства}

\section*{Источники мудрости}

% \nocite{}

\printbibliography[heading=none]


\end{document}
